\documentclass{report}
  \usepackage{amsmath}
  \usepackage{amssymb}
  \usepackage{mathrsfs}
  \usepackage{geometry}
    \geometry{letterpaper,left=1in,right=1in,top=1in,bottom=1in}
  \usepackage{bm}
  \usepackage{graphicx}
    \graphicspath{ {figures/} }
  \usepackage{float}
  \usepackage{subfiles}
  \usepackage{physics}
  \usepackage{mathtools}
  \usepackage[shortlabels]{enumitem}


  \newcommand{\dbar}{{d\mkern-7mu\mathchar'26\mkern-2mu}}

  \usepackage{xcolor}
  \usepackage{framed}
  \definecolor{shadecolor}{gray}{0.9}

  \DeclareMathOperator{\comb}{comb}
  \DeclareMathOperator{\rect}{rect}
  \DeclareMathOperator{\sinc}{sinc}

\begin{document}
\section{General solution}
  We seek a general solution to Laplace's equation on a rectangle $0\leq x \leq L$ and $0\leq y \leq H$:
  \begin{equation}
    \nabla^2 V(x,y) = 0,
  \end{equation}
  subject to
  \begin{equation}
    \begin{cases}
      V(x,0) = v_1(x) &\quad \text{(I)}\\
      V(x,H) = v_2(x) &\quad \text{(II)}\\
      V(0,y) = v_3(y) &\quad \text{(III)}\\
      V(L,y) = v_4(y) &\quad \text{(IV)}
    \end{cases}
  \end{equation}
  Because Laplace's equation is a linear equation, each boundary condition can be satisfied separately then the final solution can be superimposed.

  First, consider the boundary conditions
  \begin{equation}
    \begin{cases}
      V(x,0) = v_1(x) &\quad \text{(I)}\\
      V(x,H) = 0 &\quad \text{(II)}\\
      V(0,y) = 0 &\quad \text{(III)}\\
      V(L,y) = 0 &\quad \text{(IV)}
    \end{cases}
  \end{equation}
  Beginning with the Laplace equation in 2D
  \begin{equation}
    \nabla^2 V = 0.
  \end{equation}
  Assume a seperable solution $V(x,y) = X(x)Y(y)$. Then equation 1 becomes
  \begin{align}
    0 &= \nabla^2 X(x)Y(y)\\
    &= Y\pdv[2]{X}{x} + X\pdv[2]{Y}{y} \\
    & = \frac{1}{X}\pdv[2]{X}{x} + \frac{1}{Y}\pdv[2]{Y}{y}
  \end{align}
  Choosing a separation constant $k^2$ then the general solutions are
  \begin{align}
    \pdv[2]{X}{x} &= -k^2X\\
    \pdv[2]{Y}{y} &= k^2Y
  \end{align}
  Considering the $x$-part, the general solutions are of the form
  \begin{equation}
    X(x) = A \sin(kx) + B \cos(kx)
  \end{equation}
  Imposing the boundary conditions (III) and (IV) gives the requirement
  \begin{equation}
    X(0) = X(L) = 0
  \end{equation}
  Which gives
  \begin{equation}
    X(x) = \sum_{n=0}^\infty A_n\sin(\frac{n\pi}{L}x)
  \end{equation}

  Then the $y$-part has general solutions of the form
  \begin{equation}
    Y(y) = C \sinh(\frac{n\pi}{L} (y-H)) + D \cosh(\frac{n\pi}{L} (y-H))
  \end{equation}
  choosing the shift in $y$ by $H$ to satisfy boundary condition (II), $V(x,H) = 0$, this implies that $D=0$ as well. This gives the solution
  \begin{equation}
    Y(y) = C \sinh(\frac{n\pi}{L} (y-H))
  \end{equation}
  And total solution
  \begin{equation}
    V(x,y) = \sum_{n=0}^\infty A_n \sinh(\frac{n\pi}{L} (y-H)) \sin(\frac{n\pi}{L}x)
  \end{equation}

  To satisfy boundary condition (I), let $v_1(x,0)$ be decomposed into a fourier series
  \begin{equation}
    v_1(x,0) = \sum_{n=0}^\infty a_n \sin(\frac{n\pi}{L}x),
  \end{equation}
  where
  \begin{equation}
    a_n = \frac{2}{L} \int_0^L \dd{x} v_1(x) \sin(\frac{n\pi}{L}x).
  \end{equation}
  Which gives
  \begin{equation}
    A_n = \frac{a_n}{\sinh(-\frac{n\pi H}{L})}.
  \end{equation}

  Thus for the first set of conditions
  \begin{equation}
    \boxed{V_1(x,y) = \sum_{n=0}^\infty A_{n} \sinh(\frac{n\pi}{L} (y-H)) \sin(\frac{n\pi}{L}x)},
  \end{equation}
  with
  \begin{equation}
    \boxed{A_{n} = \frac{2/L}{\sinh(-\frac{n\pi H}{L})}  \int_0^L \dd{x} v_1(x) \sin(\frac{n\pi}{L}x)}.
  \end{equation}


  Then for the second set of boundary conditions
  \begin{equation}
    \begin{cases}
      V(x,0) = 0 &\quad \text{(I)}\\
      V(x,H) = v_2(x) &\quad \text{(II)}\\
      V(0,y) = 0 &\quad \text{(III)}\\
      V(L,y) = 0 &\quad \text{(IV)}
    \end{cases}
  \end{equation}
  \begin{equation}
    \boxed{V_2(x,y) = \sum_{n=0}^\infty B_{n} \sinh(\frac{n\pi}{L} y) \sin(\frac{n\pi}{L}x)},
  \end{equation}
  with
  \begin{equation}
    \boxed{B_{n} = \frac{2/L}{\sinh(\frac{n\pi H}{L})}  \int_0^L \dd{x} v_2(x) \sin(\frac{n\pi}{L}x)}.
  \end{equation}


  By the symmetry of $x$ and $y$ upon interchange this gives for boundary condition set
  \begin{equation}
    \begin{cases}
      V(x,0) = 0 &\quad \text{(I)}\\
      V(x,H) = 0 &\quad \text{(II)}\\
      V(0,y) = v_3(y) &\quad \text{(III)}\\
      V(L,y) = 0 &\quad \text{(IV)}
    \end{cases}
  \end{equation}
  \begin{equation}
    \boxed{V_3(x,y) = \sum_{n=0}^\infty C_{n} \sinh(\frac{n\pi}{H} (x - L)) \sin(\frac{n\pi}{H}y)},
  \end{equation}
  with
  \begin{equation}
    \boxed{C_{n} = \frac{2/H}{\sinh(-\frac{n\pi L}{H})}  \int_0^H \dd{y} v_3(y) \sin(\frac{n\pi}{H}y)}.
  \end{equation}

  And finally
  \begin{equation}
    \begin{cases}
      V(x,0) = 0 &\quad \text{(I)}\\
      V(x,H) = 0 &\quad \text{(II)}\\
      V(0,y) = 0 &\quad \text{(III)}\\
      V(L,y) = v_4(y) &\quad \text{(IV)}
    \end{cases}
  \end{equation}
  \begin{equation}
    \boxed{V_4(x,y) = \sum_{n=0}^\infty D_{n} \sinh(\frac{n\pi}{H} x) \sin(\frac{n\pi}{H}y)},
  \end{equation}
  with
  \begin{equation}
    \boxed{D_{n} = \frac{2/H}{\sinh(\frac{n\pi L}{H})}  \int_0^H \dd{y} v_4(y) \sin(\frac{n\pi}{H}y)}.
  \end{equation}

\section{Solution on a square}
  Now considering the solution to the problem on a square:
  \begin{equation}
    H \to H = L
  \end{equation}
  and
  \begin{equation}
    \begin{cases}
      V(x,0) = v_1(x) &\quad \text{(I)}\\
      V(x,L) = v_2(x) &\quad \text{(II)}\\
      V(0,y) = v_3(y) &\quad \text{(III)}\\
      V(L,y) = v_4(y) &\quad \text{(IV)}
    \end{cases}
  \end{equation}

  Then
  \begin{equation}
    \boxed{
    \begin{aligned}
      V(x,y) = \sum_{n=0}^\infty \Bigg\{\left[A_{n} \sinh(\frac{n\pi}{L} (y-L)) +
      B_{n} \sinh(\frac{n\pi}{L} y)\right] \sin(\frac{n\pi}{L}x)\\ +
      \left[C_{n} \sinh(\frac{n\pi}{L} (x - L)) +
      D_{n} \sinh(\frac{n\pi}{L} x) \right]\sin(\frac{n\pi}{L}y)  \Bigg\}
    \end{aligned}}
  \end{equation}
  where
  \begin{align}
    A_{n} &= \frac{2/L}{\sinh(-n\pi)}  \int_0^L \dd{x} v_1(x) \sin(\frac{n\pi}{L}x) \\
    B_{n} &= \frac{2/L}{\sinh(n\pi)}  \int_0^L \dd{x} v_2(x) \sin(\frac{n\pi}{L}x) \\
    C_{n} &= \frac{2/L}{\sinh(-n\pi)}  \int_0^L \dd{y} v_3(y) \sin(\frac{n\pi}{L}y) \\
    D_{n} &= \frac{2/L}{\sinh(n\pi)}  \int_0^L \dd{y} v_4(y) \sin(\frac{n\pi}{L}y).
  \end{align}



\end{document}
